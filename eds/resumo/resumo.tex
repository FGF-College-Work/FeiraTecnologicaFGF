\documentclass[12pt]{article}
\usepackage{graphicx,url}
\usepackage[brazil]{babel}   
%\usepackage[latin1]{inputenc}  
\usepackage[utf8]{inputenc}  %permite texto com acento
% UTF-8 encoding is recommended by ShareLaTex
\usepackage{verbatim}
\usepackage{listings}
\usepackage{xcolor}
\usepackage[margin=2.5cm]{geometry}
\usepackage{xcolor}
\usepackage{fancyhdr}
\pagestyle{fancy}
\fancyhf{}
\rfoot{Página \thepage}
\lhead{\it{\textcolor{gray}{\small II Mostra Científica de Tecnologia - FGF}}}
\rhead{\it{\textcolor{gray}{\small 02, 03 e 04 de Março de 2017}}}
\chead{\it{\textcolor{gray}{\small Fortaleza - Ce}}}
\lfoot{\it{X Feira Tecnológica - FGF}}
\cfoot{\it{Fortaleza - Ceará}}
\renewcommand{\headrulewidth}{0.5pt}

%%%%%%%%%%%%%%%%%%%%%%%%%%%%

\begin{document}
%\onehalfspacing

%
%%% Não alterar o preâmbulo acima.

    %%% MARQUE UM X NO TIPO DE PESQUISA

\noindent{TIPO DE RESUMO: 1. Trabalho original( ), 2. Relato de experiência( ), 3.Estudo de caso( ), 4. Pesquisa bibliométrica( ), 5. Pesquisa bibliográfica ( ), 6. Reflexão crítica( ), 7. Relatório técnico (x), 8. Trabalho de conclusão de curso( ), 9. Nota prévia de monografia( ), 10. Relatório final de monografia( ).\\ASSINALE O TIPO DE APRESENTAÇÃO: 1. ORAL(X) 2. POSTER( ).
}
%%%%%%%%%%% TÍTULO %%%%%%%%%%%

\begin{center}
\textbf{\Large{Um Estudo de Benchmark das Tecnologias Web implementadas em Portais Acadêmicos de Faculdades no Estado do Ceará}}\\
\end{center}

\vspace*{0.2cm}
%%%%%%%%%%% AUTORES - NO MÁXIMO 4 E 1 ORIENTADOR - %%%%%%%%%%%

\begin{flushright}
 {\bf Cleison de Sousa Pereira} \footnote[1]{Graduando em Sistemas para Internet - FGF. e-mail: \it cleilsonpereira@aluno.fgf.edu.br}  \\
 {\bf Nome Completo do Aluno} \footnote[2]{Graduando em Ciência da Computação - FGF. e-mail: \it aluno@aluno.fgf.edu.br}  \\
  {\bf Nome Completo do Aluno} \footnote[3]{Graduando em Ciência da Computação - FGF. e-mail: \it aluno@aluno.fgf.edu.br}  \\
   {\bf Rute Nogueira Silveira de Castro} \footnote[4]{Mestre - Faculdade da Grande Fortaleza. e-mail: \it rute@fgf.edu.br}   \\
\end{flushright}

\vspace*{0.5cm}

%%%%%%%%%%% CORPO DO TRABALHO - ENTRE 200 E 600 PALAVRAS %%%%%%%%%%%
%%%% TODOS OS TRABALHOS DEVEM TRAZER AS SEÇÕES EXATAMENTE COMO DESTACADO ABAIXO %%%%%%


\noindent{\textbf{INTRODUÇÃO:} O \emph{benchmark} é utilizado para verificar e comparar as melhores práticas de negócio, baseado nas metricas de desempenho e processos de negócio entre empresas, para direcionar uma estrategia de ação no planejamento, apoio, seleção e entrega de projetos. \textbf{OBJETIVO:}Este estudo apresenta uma analise abrangente sobre a adoção de tecnologias web nas principais faculdade e universidade de Fortaleza. Com o intuitio de apontar uma tendência padrão no desenvolvimento para esse nicho de mercado, permitindo um reconhecimento estrategico para validaçao de um futuro investimento.\textbf{METODOLOGIA:} A metodologia de benchmark utilizada foi o estudo do benchmark de tecnologia de informação e aplicativo. Abrangendo aspectos do processamento dos dados, analise do sistema, desenvolvimento, programação e usabilidade. O presente estudo se limita a analisar o desenvolvimento e programação, se valendo de ferramentas online de verificação de trafego e tendência tecnologica para estabelecer uma base de dados para o presente estudo.\textbf{RESULTADOS E DISCUSSÃO:} XXX XXX XXX XXX XXX XXX XXX XXX XXX XXX XXX XXX XXX XXX XXX XXX XXX XXX XXX XXX XXXXXX XXX XXX XXX XXX XXX XXX XXX XXX XXX XXX XXX XXX XXX XXX XXX XXX XXX XXX XXX XXXXXX XXX XXX XXX XXX XXX. \textbf{CONCLUSÃO:} XXX XXX XXX XXX XXX XXX XXX XXX XXX XXX XXXXXX XXX XXX XXX XXX XXX XXX XXX XXX XXX XXX XXX XXX XXX XXX XXX XXX XXX XXX XXX XXXXXX XXX XXX XXX XXX XXX XXX XXX XXX XXX XXX XXX XXX XXX XXX XXX XXX XXX. \textbf{REFERÊNCIAS BIBLIOGRÁFICAS (MAX. TRÊS):} XXX XXX XXX XXX XXX XXX XXX XXX XXX XXX XXX XXXXXX XXX XXX XXX XXX XXX XXX XXX XXX XXX XXX XXXXXXXXXXXXXXXX XXX .XXXXXXXXXXXXXXXXXXXXXX}

%%%%%%%%%%% FIM DO RESUMO %%%%%%%%%%%
%%%%%%%%%%% O RESUMO DEVE TER NO MÁXIMO DUAS PÁGINAS %%%%%%%%%%%


\end{document}


