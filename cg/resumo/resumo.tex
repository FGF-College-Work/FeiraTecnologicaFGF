\documentclass[12pt]{article}
\usepackage{graphicx,url}
\usepackage[brazil]{babel}   
%\usepackage[latin1]{inputenc}  
\usepackage[utf8]{inputenc}  %permite texto com acento
% UTF-8 encoding is recommended by ShareLaTex
\usepackage{verbatim}
\usepackage{listings}
\usepackage{xcolor}
\usepackage[margin=2.5cm]{geometry}
\usepackage{xcolor}
\usepackage{fancyhdr}
\pagestyle{fancy}
\fancyhf{}
\rfoot{Página \thepage}
\lhead{\it{\textcolor{gray}{\small II Mostra Científica de Tecnologia - FGF}}}
\rhead{\it{\textcolor{gray}{\small 02, 03 e 04 de Março de 2017}}}
\chead{\it{\textcolor{gray}{\small Fortaleza - Ce}}}
\lfoot{\it{X Feira Tecnológica - FGF}}
\cfoot{\it{Fortaleza - Ceará}}
\renewcommand{\headrulewidth}{0.5pt}

%%%%%%%%%%%%%%%%%%%%%%%%%%%%

\begin{document}
%\onehalfspacing

%
%%% Não alterar o preâmbulo acima.

    %%% MARQUE UM X NO TIPO DE PESQUISA

\noindent{TIPO DE RESUMO: 1. Trabalho original( ), 2. Relato de experiência( ), 3.Estudo de caso( ), 4. Pesquisa bibliométrica( ), 5. Pesquisa bibliográfica (X), 6. Reflexão crítica( ), 7. Relatório técnico ( ), 8. Trabalho de conclusão de curso( ), 9. Nota prévia de monografia( ), 10. Relatório final de monografia( ).\\ASSINALE O TIPO DE APRESENTAÇÃO: 1. ORAL(X) 2. POSTER( ).
}
%%%%%%%%%%% TÍTULO %%%%%%%%%%%

\begin{center}
\textbf{\Large{UMA VISÃO GERAL SOBRE COORDENADAS HOMOGÊNEAS APLICADAS NA COMPUTAÇÃO GRÁFICA}}\\
\end{center}

\vspace*{0.2cm}
%%%%%%%%%%% AUTORES - NO MÁXIMO 4 E 1 ORIENTADOR - %%%%%%%%%%%

\begin{flushright}
 {\bf Ceilson de Sousa Pereira} \footnote[1]{Graduando em Sistemas para Internet - FGF. e-mail: \it aluno@aluno.fgf.edu.br}  \\
 {\bf Eder Sousa} \footnote[2]{Graduando em Sistemas para Internet - FGF. e-mail: \it aluno@aluno.fgf.edu.br}  \\
  {\bf Carla Adrielle Diogo} \footnote[3]{Graduando em Sistemas para Internet - FGF. e-mail: \it aluno@aluno.fgf.edu.br}  \\
   {\bf Hitalo Joseferson Batista Nascimento} \footnote[4]{Mestre - Faculdade da Grande Fortaleza. e-mail: \it hitalo@fgf.edu.br}   \\
\end{flushright}

\vspace*{0.5cm}

%%%%%%%%%%% CORPO DO TRABALHO - ENTRE 200 E 600 PALAVRAS %%%%%%%%%%%
%%%% TODOS OS TRABALHOS DEVEM TRAZER AS SEÇÕES EXATAMENTE COMO DESTACADO ABAIXO %%%%%%


\noindent{\textbf{INTRODUÇÃO:} Em matemática, coordenadas homogêneas ou coordenadas projetivas, são um sistema de coordenadas utilizadas na geometria projetiva. As coordenadas homogêneas são onipresentes em gráficos computacionais porque permitem que operações de vetores comuns, como a tradução, a rotação, a escala e a projeção de perspectiva,  para serem representadas como uma matriz pela qual o vetor é multiplicado. Aonde todas essas operações matemáticas são utilizadas em softwares especifico como openGL e Direct3D. As coordenadas homogêneas possuem uma gama de aplicações, incluindo graficos computacionais e visão computacional 3D. \textbf{OBJETIVO:} Apresentar um panorama da utilização aplicada das coordenadas homogêneas na computação gráfica e os principais softwares e bibliotecas para geração e transformação de figuras geometricas. E uma análise da produção acadêmica nesse aspecto especifico, dentro da computação gráfica, está abordado no presente estudo.\textbf{METODOLOGIA:} A metodologia utilizada no presente estudo foi a pesquisa bibliográfica. Realizando pesquisa em periodicos acadêmicos cientificos como Google Acadêmico, CAPES, SciELO, Spell e Microsoft Academic. \textbf{RESULTADOS E DISCUSSÃO:} A pesquisa foi delimitada no escopo de tempo referente ao ano de 2018. Com intuito de abordar os resultados obtidos na pesquisa, foi encontrado pesquisas na área de visão computacional, aplicação dos fundamentos matemáticos de coordenadas homogêneas na calibração de camera, e computação gráfica educacional. Listando ferramentas como openGL, Geomview e Direct3D.  \textbf{CONCLUSÃO:} O panorama proposto pelo presente estudo, mostrou como a base matemática está intrisicamente conectada a cada aspecto da computação gráfica. Deixando para estudos posteriores a análise prática de sua aplicação, no uso geral das ferramentas citadas. \textbf{REFERÊNCIAS BIBLIOGRÁFICAS (MAX. TRÊS):} SHAO, Xi. XU, Changsheng. LIM, Joo Hwee.\textbf{Image mosaics base on homogeneous coordinates}.2003. SCHEIDER, Philip. EBERLY, David H.\textbf{Geometric Tools for Computer Graphics}.2003}

%%%%%%%%%%% FIM DO RESUMO %%%%%%%%%%%
%%%%%%%%%%% O RESUMO DEVE TER NO MÁXIMO DUAS PÁGINAS %%%%%%%%%%%


\end{document}


