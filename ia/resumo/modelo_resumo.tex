\documentclass[12pt]{article}
\usepackage{graphicx,url}
\usepackage[brazil]{babel}   
%\usepackage[latin1]{inputenc}  
\usepackage[utf8]{inputenc}  %permite texto com acento
% UTF-8 encoding is recommended by ShareLaTex
\usepackage{verbatim}
\usepackage{listings}
\usepackage{xcolor}
\usepackage[margin=2.5cm]{geometry}
\usepackage{xcolor}
\usepackage{fancyhdr}
\pagestyle{fancy}
\fancyhf{}
\rfoot{Página \thepage}
\lhead{\it{\textcolor{gray}{\small II Mostra Científica de Tecnologia - FGF}}}
\rhead{\it{\textcolor{gray}{\small 02, 03 e 04 de Março de 2017}}}
\chead{\it{\textcolor{gray}{\small Fortaleza - Ce}}}
\lfoot{\it{X Feira Tecnológica - FGF}}
\cfoot{\it{Fortaleza - Ceará}}
\renewcommand{\headrulewidth}{0.5pt}

%%%%%%%%%%%%%%%%%%%%%%%%%%%%

\begin{document}
%\onehalfspacing

%
%%% Não alterar o preâmbulo acima.

    %%% MARQUE UM X NO TIPO DE PESQUISA

\noindent{TIPO DE RESUMO: 1. Trabalho original( ), 2. Relato de experiência( ), 3.Estudo de caso( ), 4. Pesquisa bibliométrica(X), 5. Pesquisa bibliográfica ( ), 6. Reflexão crítica( ), 7. Relatório técnico ( ), 8. Trabalho de conclusão de curso( ), 9. Nota prévia de monografia( ), 10. Relatório final de monografia( ).\\ASSINALE O TIPO DE APRESENTAÇÃO: 1. ORAL(X) 2. POSTER( ).
}
%%%%%%%%%%% TÍTULO %%%%%%%%%%%

\begin{center}
\textbf{\Large{TITULO DO TRABALHO}}\\
\end{center}

\vspace*{0.2cm}
%%%%%%%%%%% AUTORES - NO MÁXIMO 4 E 1 ORIENTADOR - %%%%%%%%%%%

\begin{flushright}
 {\bf Nome Completo do Aluno} \footnote[1]{Graduando em Ciência da Computação - FGF. e-mail: \it aluno@aluno.fgf.edu.br}  \\
 {\bf Nome Completo do Aluno} \footnote[2]{Graduando em Ciência da Computação - FGF. e-mail: \it aluno@aluno.fgf.edu.br}  \\
  {\bf Nome Completo do Aluno} \footnote[3]{Graduando em Ciência da Computação - FGF. e-mail: \it aluno@aluno.fgf.edu.br}  \\
   {\bf Elidiane Martins Freitas} \footnote[4]{Mestre - Faculdade da Grande Fortaleza. e-mail: \it elidiane@fgf.edu.br}   \\
\end{flushright}

\vspace*{0.5cm}

%%%%%%%%%%% CORPO DO TRABALHO - ENTRE 200 E 600 PALAVRAS %%%%%%%%%%%
%%%% TODOS OS TRABALHOS DEVEM TRAZER AS SEÇÕES EXATAMENTE COMO DESTACADO ABAIXO %%%%%%


\noindent{\textbf{INTRODUÇÃO:} XXX XXX XXX XXX XXX XXX XXX XXX XXX XXX XXX XXX XXXXXX XXX XXX XXX XXX XXX XXX XXX XXX XXX XXX XXX XXX XXX XXX XXX XXX XXX XXXXXX XXX. \textbf{OBJETIVO:} XXX XXX XXX XXX XXX XXX XXX XXX XXX XXX XXX XXX XXXXXX XXX XXX XXX XXX XXX XXX XXX XXX XXX XXX XXX XXX XXX XXX XXX XXX XXX XXXXXX XXX. \textbf{METODOLOGIA:} XXX XXX XXX XXX XXX XXX XXX XXX XXX XXX XXX XXX XXX XXX XXX XXXXXX XXX XXX XXX XXX XXX XXX XXX XXX XXX XXX XXX XXX XXX XXX XXX XXX XXX XXX XXX XXXXXX XXX XXX XXX XXX XXX XXX XXX XXX XXX XXX XXX XXX XXX XXX XXX XXX. \textbf{RESULTADOS E DISCUSSÃO:} XXX XXX XXX XXX XXX XXX XXX XXX XXX XXX XXX XXX XXX XXX XXX XXX XXX XXX XXX XXX XXXXXX XXX XXX XXX XXX XXX XXX XXX XXX XXX XXX XXX XXX XXX XXX XXX XXX XXX XXX XXX XXXXXX XXX XXX XXX XXX XXX. \textbf{CONCLUSÃO:} XXX XXX XXX XXX XXX XXX XXX XXX XXX XXX XXXXXX XXX XXX XXX XXX XXX XXX XXX XXX XXX XXX XXX XXX XXX XXX XXX XXX XXX XXX XXX XXXXXX XXX XXX XXX XXX XXX XXX XXX XXX XXX XXX XXX XXX XXX XXX XXX XXX XXX. \textbf{REFERÊNCIAS BIBLIOGRÁFICAS (MAX. TRÊS):} XXX XXX XXX XXX XXX XXX XXX XXX XXX XXX XXX XXXXXX XXX XXX XXX XXX XXX XXX XXX XXX XXX XXX XXXXXXXXXXXXXXXX XXX .XXXXXXXXXXXXXXXXXXXXXX}

%%%%%%%%%%% FIM DO RESUMO %%%%%%%%%%%
%%%%%%%%%%% O RESUMO DEVE TER NO MÁXIMO DUAS PÁGINAS %%%%%%%%%%%


\end{document}


