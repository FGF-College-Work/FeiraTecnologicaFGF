\documentclass[12pt]{article}
\usepackage{graphicx,url}
\usepackage[brazil]{babel}   
%\usepackage[latin1]{inputenc}  
\usepackage[utf8]{inputenc}  %permite texto com acento
% UTF-8 encoding is recommended by ShareLaTex
\usepackage{verbatim}
\usepackage{listings}
\usepackage{xcolor}
\usepackage[margin=2.5cm]{geometry}
\usepackage{xcolor}
\usepackage{fancyhdr}
\pagestyle{fancy}
\fancyhf{}
\rfoot{Página \thepage}
\lhead{\it{\textcolor{gray}{\small II Mostra Científica de Tecnologia - FGF}}}
\rhead{\it{\textcolor{gray}{\small 02, 03 e 04 de Março de 2017}}}
\chead{\it{\textcolor{gray}{\small Fortaleza - Ce}}}
\lfoot{\it{X Feira Tecnológica - FGF}}
\cfoot{\it{Fortaleza - Ceará}}
\renewcommand{\headrulewidth}{0.5pt}

%%%%%%%%%%%%%%%%%%%%%%%%%%%%

\begin{document}
%\onehalfspacing

%
%%% Não alterar o preâmbulo acima.

    %%% MARQUE UM X NO TIPO DE PESQUISA

\noindent{TIPO DE RESUMO: 1. Trabalho original( ), 2. Relato de experiência( ), 3.Estudo de caso( ), 4. Pesquisa bibliométrica( ), 5. Pesquisa bibliográfica (X), 6. Reflexão crítica( ), 7. Relatório técnico ( ), 8. Trabalho de conclusão de curso( ), 9. Nota prévia de monografia( ), 10. Relatório final de monografia( ).\\ASSINALE O TIPO DE APRESENTAÇÃO: 1. ORAL(X) 2. POSTER( ).
}
%%%%%%%%%%% TÍTULO %%%%%%%%%%%

\begin{center}
\textbf{\Large{UM ESTUDO SISTEMÁTICO SOBRE O USO DE AGENTES INTELIGENTES EM MOTORES E SISTEMAS DE BUSCA WEB}}\\
\end{center}

\vspace*{0.2cm}
%%%%%%%%%%% AUTORES - NO MÁXIMO 4 E 1 ORIENTADOR - %%%%%%%%%%%

\begin{flushright}
 {\bf Cleilson de Sousa Pereira} \footnote[1]{Graduando em Sistemas para Internet - FGF. e-mail: \it aluno@aluno.fgf.edu.br}  \\
 {\bf Eder Sousa} \footnote[2]{Graduando em Sistemas para Internet - FGF. e-mail: \it aluno@aluno.fgf.edu.br}  \\
  {\bf Dione} \footnote[3]{Graduando em Sistemas para Internet - FGF. e-mail: \it aluno@aluno.fgf.edu.br}  \\
   {\bf Rafael Teixeira Araujo} \footnote[4]{Mestre - Faculdade da Grande Fortaleza. e-mail: \it rafael@fgf.edu.br}   \\
\end{flushright}

\vspace*{0.5cm}

%%%%%%%%%%% CORPO DO TRABALHO - ENTRE 200 E 600 PALAVRAS %%%%%%%%%%%
%%%% TODOS OS TRABALHOS DEVEM TRAZER AS SEÇÕES EXATAMENTE COMO DESTACADO ABAIXO %%%%%%


\noindent{\textbf{INTRODUÇÃO:} Em inteligência artificial, um agente inteligente é uma entidade autônoma que observa atráves de sensores e atua em um ambiente usando atuadores, ele orienta sua ativiade para alcançar um objetivo. O agente inteligente é quem faz a escolha da melhor ação possivel para um problema proposto, está hoje imbutido no cotidiano dos usuarios da \textit{internet}, desde buscas complexas de comparativo de preços até em uma simples pesquisa de manual de celular \textbf{OBJETIVO:} O presente estudo visa elaborar e contextualizar o uso de aplicações de agentes inteligentes,com foco em sistemas e motores de pesquisa \textit{web}. Direcionando a pesquisa para uma listagem de segmentos estratégicos, na tomada de decisão em diferentes negócios, baseado na coleta e direcionamento da inteligência, construida a partir da informação coletada. \textbf{METODOLOGIA:}A metodologia aplicada no presente estudo é a pesquisa bibliográfica. Analisando a literatura e produção acadêmica, nesse segmento de desenvolvimento e aplicação de inteligência artificial. Pesquisado periodicos acadêmicos cientificos como \textit{Google Acadêmico, CAPES, Spell, SciELO e Microsoft Academic}. \textbf{RESULTADOS E DISCUSSÃO:} O presente estudo teve como resultado, um retorno de diversas implementações da inteligência artificial nos mais variados segmentos setoriais. Desde o uso de recomendações baseado no conteudo pesquisado, à trabalhos no gerenciamentos de informações coletadas pelos agentes inteligentes. O gerenciamento de tomada de decisão, está cada dia mais conectado, direta ou indiretamente, ao resultado do uso de agentes inteligentes. \textbf{CONCLUSÃO:} Como conclusão, o estudo apresentado apontou uma tendência tecnologica, de implementação geral no desenvolvimento de aplicações, baseada em sistemas de busca na internet. Como um apoio estratégico do negócio, e direcioandor na tomada de decisões. \textbf{REFERÊNCIAS BIBLIOGRÁFICAS:} GAEBLER,Rafael.\textbf{Agentes inteligentes para pesquisas na Internet}.2004, SILVA, Helena Pereira da.\textbf{Inteligência competitiva na Internet:}um processo otimizado por agentes inteligentes.2003, FERREIRA,Fábio Antônio.\textbf{Aplicação de inteligência artificial utilizando agentes inteligentes – análise de um jogo educativo na web}.2010}

%%%%%%%%%%% FIM DO RESUMO %%%%%%%%%%%
%%%%%%%%%%% O RESUMO DEVE TER NO MÁXIMO DUAS PÁGINAS %%%%%%%%%%%


\end{document}


