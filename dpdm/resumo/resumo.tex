\documentclass[12pt]{article}
\usepackage{graphicx,url}
\usepackage[brazil]{babel}   
%\usepackage[latin1]{inputenc}  
\usepackage[utf8]{inputenc}  %permite texto com acento
% UTF-8 encoding is recommended by ShareLaTex
\usepackage{verbatim}
\usepackage{listings}
\usepackage{xcolor}
\usepackage[margin=2.5cm]{geometry}
\usepackage{xcolor}
\usepackage{fancyhdr}
\pagestyle{fancy}
\fancyhf{}
\rfoot{Página \thepage}
\lhead{\it{\textcolor{gray}{\small II Mostra Científica de Tecnologia - FGF}}}
\rhead{\it{\textcolor{gray}{\small 02, 03 e 04 de Março de 2017}}}
\chead{\it{\textcolor{gray}{\small Fortaleza - Ce}}}
\lfoot{\it{X Feira Tecnológica - FGF}}
\cfoot{\it{Fortaleza - Ceará}}
\renewcommand{\headrulewidth}{0.5pt}

%%%%%%%%%%%%%%%%%%%%%%%%%%%%

\begin{document}
%\onehalfspacing

%
%%% Não alterar o preâmbulo acima.

    %%% MARQUE UM X NO TIPO DE PESQUISA

\noindent{TIPO DE RESUMO: 1. Trabalho original( ), 2. Relato de experiência( ), 3.Estudo de caso( ), 4. Pesquisa bibliométrica(X), 5. Pesquisa bibliográfica ( ), 6. Reflexão crítica( ), 7. Relatório técnico ( ), 8. Trabalho de conclusão de curso( ), 9. Nota prévia de monografia( ), 10. Relatório final de monografia( ).\\ASSINALE O TIPO DE APRESENTAÇÃO: 1. ORAL(X) 2. POSTER( ).
}
%%%%%%%%%%% TÍTULO %%%%%%%%%%%

\begin{center}
\textbf{\Large{UM ESTUDO ANALITICO SOBRE O USO E APLICAÇÃO DE RECONHECIMENTO DE PADRÕES EM DISPOSITIVOS MÓVEIS}}\\
\end{center}

\vspace*{0.2cm}
%%%%%%%%%%% AUTORES - NO MÁXIMO 4 E 1 ORIENTADOR - %%%%%%%%%%%

\begin{flushright}
 {\bf Cleilson de Sousa Pereira} \footnote[1]{Graduando em Sistemas para Internet - FGF. e-mail: \it cleilsonpereira@aluno.fgf.edu.br}  \\
 {\bf Eder Sousa} \footnote[2]{Graduando em Sistemas para Internet - FGF. e-mail: \it edersousa@aluno.fgf.edu.br}  \\
  {\bf Claudia Adrielle Diogo} \footnote[3]{Graduando em Sistemas para Internet - FGF. e-mail: \it adrielle@aluno.fgf.edu.br}  \\
   {\bf Hitalo Joseferson Batista Nascimento} \footnote[4]{Mestre - Faculdade da Grande Fortaleza. e-mail: \it hitalo@fgf.edu.br}   \\
\end{flushright}

\vspace*{0.5cm}

%%%%%%%%%%% CORPO DO TRABALHO - ENTRE 200 E 600 PALAVRAS %%%%%%%%%%%
%%%% TODOS OS TRABALHOS DEVEM TRAZER AS SEÇÕES EXATAMENTE COMO DESTACADO ABAIXO %%%%%%


\noindent{\textbf{INTRODUÇÃO:} O reconhecimento de padrões é uma ramificação do aprendizado de maquina com foco na averiguação dos padrões e regularidades dos dados analisados. O reconhecimento de padrões é geralmente categorizado de acordo com o tipo de procedimento de aprendizagem usado para gerar o valor de saída, sendo utilizado a apredizagem supervisionada e a apredizagem não supervisonada. A aprendizagem supervisionada requer uma base da dados de treinamento e a aprendizagem não supervisonada irá averiguar similaridades nos dados sem uma base de treinamento, encontrando padrões e agrupando os dados por ordem de similaridade.\textbf{OBJETIVO:} Apontar as principais aplicações de algoritmos de reconhecimento de padrões, em implementações no desenvolvimento de dispositivos móveis. E analisar o seguimento de negócio no desenvolvimento, baseado na literatura acadêmica pequisada. \textbf{METODOLOGIA:} A metodologia aplicada ao seguinte estudo foi a pesquisa bibliografica. Feito análise de produção escrita em periodicos científicos, como CAPES, SciElo, Spell, Google Acadêmico e Microsoft Academic. Material complementar baseado em estudo bibliométrico, para apontamento de métricas, e tendências foi introduzido no presente estudo também.\textbf{RESULTADOS E DISCUSSÃO:} Com base preliminar da literatura acadêmica pesquisada para o estudo em questão, as aplicações para reconhecimento de padrões mais encotradas estão na visão computacional, reconhecimento de voz, classificação de texto por categoria e no diagnostico assistido por computador entre outros. O direcionamento e uso prático no desenvolvimento de aplicações móveis, conforme resultado da seguinte pesquisa para este estudo, está na implementação de metodos de segurança como biometria, faceID e ferramentas de apoio a decisão na telemedicina, no processamento de imagem da radiologia.\textbf{CONCLUSÃO:} A inteligêcia artificial e suas ramificações se apresetam cada dia mais presente, nas mais diversas aréa de atuação multidisciplinares. Automatizando e tratando a informação para geração de conhecimento apicado no negócio e aliado ao desenvolviemnto móvel de forma estratégica consegue propor soluções portáteis e mais usuais.\textbf{REFERÊNCIAS BIBLIOGRÁFICAS:} NAGY, George.\textbf{Interactive, Mobile, Distributed Pattern Recognition}.2005, FERNANDES,Esteban Vazquez, JIMENEZ,Daniel Gonzalez.\textbf{Face recognition for authentication on mobile devices.}2016.Gradiant, KHARE,Shivam.\textbf{Finger gesture and pattern recognition based device security system.}2015.IEEE}

%%%%%%%%%%% FIM DO RESUMO %%%%%%%%%%%
%%%%%%%%%%% O RESUMO DEVE TER NO MÁXIMO DUAS PÁGINAS %%%%%%%%%%%


\end{document}


