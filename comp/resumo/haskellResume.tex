\documentclass[12pt]{article}
\usepackage{graphicx,url}
\usepackage[brazil]{babel}   
%\usepackage[latin1]{inputenc}  
\usepackage[utf8]{inputenc}  %permite texto com acento
% UTF-8 encoding is recommended by ShareLaTex
\usepackage{verbatim}
\usepackage{listings}
\usepackage{xcolor}
\usepackage[margin=2.5cm]{geometry}
\usepackage{xcolor}
\usepackage{fancyhdr}
\pagestyle{fancy}
\fancyhf{}
\rfoot{Página \thepage}
\lhead{\it{\textcolor{gray}{\small II Mostra Científica de Tecnologia - FGF}}}
\rhead{\it{\textcolor{gray}{\small 02, 03 e 04 de Março de 2017}}}
\chead{\it{\textcolor{gray}{\small Fortaleza - Ce}}}
\lfoot{\it{X Feira Tecnológica - FGF}}
\cfoot{\it{Fortaleza - Ceará}}
\renewcommand{\headrulewidth}{0.5pt}

%%%%%%%%%%%%%%%%%%%%%%%%%%%%

\begin{document}
%\onehalfspacing

%
%%% Não alterar o preâmbulo acima.

    %%% MARQUE UM X NO TIPO DE PESQUISA

\noindent{TIPO DE RESUMO: 1. Trabalho original( ), 2. Relato de experiência( ), 3.Estudo de caso( ), 4. Pesquisa bibliométrica( ), 5. Pesquisa bibliográfica (X), 6. Reflexão crítica( ), 7. Relatório técnico ( ), 8. Trabalho de conclusão de curso( ), 9. Nota prévia de monografia( ), 10. Relatório final de monografia( ).\\ASSINALE O TIPO DE APRESENTAÇÃO: 1. ORAL(X) 2. POSTER( ).
}
%%%%%%%%%%% TÍTULO %%%%%%%%%%%

\begin{center}
\textbf{\Large{UMA VISÃO GERAL DO USO E APLICAÇÃO DA LINGUAGEM DE PROGRAMAÇÃO HASKELL NO DESENVOLVIMENTO}}\\
\end{center}

\vspace*{0.2cm}
%%%%%%%%%%% AUTORES - NO MÁXIMO 4 E 1 ORIENTADOR - %%%%%%%%%%%

\begin{flushright}
 {\bf Cleilson de Sousa Pereira} \footnote[1]{Graduando em Sistemas para Internet - FGF. e-mail: \it cleilsonpereira@aluno.fgf.edu.br}  \\
 {\bf Antonio Adail de Sousa Júnior} \footnote[2]{Graduando em Ciência da Computação - FGF. e-mail: \it adail.si579@hotmail.com}  \\
  {\bf Nome Completo do Aluno} \footnote[3]{Graduando em Ciência da Computação - FGF. e-mail: \it aluno@aluno.fgf.edu.br}  \\
   {\bf Rafael Teixeira de Araújo} \footnote[4]{Mestre - Faculdade da Grande Fortaleza. e-mail: \it rafael@fgf.edu.br}   \\
\end{flushright}

\vspace*{0.5cm}

%%%%%%%%%%% CORPO DO TRABALHO - ENTRE 200 E 600 PALAVRAS %%%%%%%%%%%
%%%% TODOS OS TRABALHOS DEVEM TRAZER AS SEÇÕES EXATAMENTE COMO DESTACADO ABAIXO %%%%%%


\noindent{\textbf{INTRODUÇÃO:} A linguagem de programação Haskell, é uma linguagem puramente funcional,tipagem estática, sintaxe similar a notação matemática. Muito utilizada no meio acadêmico, na parte de pesquisa, mas tambem serviu de base de pesquisa para desenvolviemento de trabalhos em mercados comerciais, como a linguagem Bluespec SystemVerilog, desenvolvida como uma extensão da linguagem haskell. A linguagem ainda permite uma forte integração e suporte a outras linguagens. \textbf{OBJETIVO:} O intuito desse estudo é apresentar as aplicações modernas da linguagem, na estrutura de desenvolvimento de tecnologias da informação. Listando principais aplicações e empresas que utilizam a linguagem haskell em sua stack de desenvolvimento. Objetivando apresentar uma analise de usabilidade da linguagem, apesar de baixa popularidade da linguagem como escolha principal de programadores. \textbf{METODOLOGIA:} Para a realização desta pesquisa foram utilizadas informações documentais, aplicando a pesquisa bibliográfica como metodologia de estudo. Periodicos acadêmicos e cientificos pesquisados no presente estudo foram \textit{Google Acadêmico, SciELO, Spell, CAPES, Microsoft Academic} e o site oficial da linguagem haskell. Como dados complemetares foi analisado metricas de tendências de pesquisa utilizando a ferramenta \textit{Google Trends}\textbf{CONCLUSÃO:} Como conclusão do presente estudo, diversas tecnologia apropriadas para o desenvolvimento foram listadas e apresentadas como aplicações para desevolvimento web, mobile e geradores de site estáticos. Apesar da tendência de novas linguagens especificas para algum nicho de mercado. Haskell ainda é utilizado devido a eficiência com dados matemáticos. \textbf{REFERÊNCIAS BIBLIOGRÁFICAS:} SIMON,Peyton Jones.\textbf{A History of Haskell: being lazy with class}.2007, SNOYMAN,Michael.\textbf{Developing web applications with Haskell and Yesod}.2012, SNOYMAN,Michael.\textbf{Warp: A Haskell Web Server}.2011}

%%%%%%%%%%% FIM DO RESUMO %%%%%%%%%%%
%%%%%%%%%%% O RESUMO DEVE TER NO MÁXIMO DUAS PÁGINAS %%%%%%%%%%%


\end{document}


